\documentclass{VUMIFPSkursinis}
\usepackage{algorithmicx}
\usepackage{algorithm}
\usepackage{algpseudocode}
\usepackage{amsfonts}
\usepackage{amsmath}
\usepackage{bm}
\usepackage{caption}
\usepackage{color}
\usepackage{float}
\usepackage{graphicx}
\usepackage{listings}
\usepackage{subfig}
\usepackage{wrapfig}
% \usepackage{lithuanian}
\usepackage{longtable}

\usepackage{enumitem}
%PAKEISTA, tarpai tarp sąrašo elementų
\setitemize{noitemsep,topsep=0pt,parsep=0pt,partopsep=0pt}
\setenumerate{noitemsep,topsep=0pt,parsep=0pt,partopsep=0pt}

% Titulinio aprašas
\university{Vilniaus universitetas}
\faculty{Matematikos ir informatikos fakultetas}
\department{Programų sistemų katedra}
\papertype{Kursinis darbas}
\title{Rodiklių duomenų kaupimas, transformavimas ir analizė, naudojant NoSQL duomenų bazę}
\titleineng{(Storage, transformation and analysis of indicator data with the help of NoSQL database)}
\status{3 kurso 5 grupės studentas}
\author{Vytautas Žilinas}
\supervisor{lekt. Andrius Adamonis}
\date{Vilnius – \the\year}

% Nustatymai
% \setmainfont{Palemonas}   % Pakeisti teksto šriftą į Palemonas (turi būti įdiegtas sistemoje)
%\bibliography{bibliografija}
\documentclass{article}
\usepackage[backend=biber]{biblatex}
\addbibresource{bibliografija.bib}
\begin{document}
	
% PAKEISTA	
\maketitle
\cleardoublepage\pagenumbering{arabic}
\setcounter{page}{2}

%TURINYS
\tableofcontents

\sectionnonum{Įvadas}

Darbo tikslas: Eksperimento būdu išbandyti rodiklių duomenų kaupimo, transformavimo ir analizės sprendimus, palyginant sprendimą, 
naudojanti reliacinę duomenų bazę, su sprendimu naudojančiu srautinį duomenų apdorojimą.

Užduotys:
\begin{enumerate}
    \item Analizės budu palyginti srautinį ir reliacinį duomenų apdorojimo sprendimą.
    \item Sukurti testiniu duomenų generatorių.
    \item Išmatuoti reliacinės duomenų bazės sprendimo pralaidumą.
    \item Atlikti skirtingų srautinio duomenų apdorojimo sprendimo architektūrų analizę, ir pasirinkti vieną iš jų tyrimui.
    \item Išmatuoti pasirinktos srautinio duomenų apdorojimo architektūros sprendimo pralaidumą.
\end{enumerate}


\section{Rodiklių duomenys}

\subsection{Apibrėžimas}

Rodiklių duomenys - tai didelių duomenų tipas, kurį galima transformuoti ir analizuoti ir kuris yra sugrupuotas pagal rodiklius, 
pavyzdžiui: bazinė menėsio alga, mirusiųjų skaičius pagal mirties priežastis, krituliai per metus. Šie duomenys dažniausiai yra saugomi reliacinėse duomenų bazėse, 
kur užklausus vartotojui skaičiuojami apibendrinti rodikliai - sumos, vidurkiai ir kita statistika.
Lietuvoje pagrindinis rodiklių duomenų bazės pavyzdys yra ,,Lietuvos statistikos departamento'' duomenų bazė, kurios duomenis galima pasiekti
\url{https://osp.stat.gov.lt/statistiniu-rodikliu-analize#/} puslapyje, kuris leidžia ieškoti duomenis pagal vieną arba kelis rodiklius. Didesnis pavyzdys yra ,,DataBank''
\url{http://databank.worldbank.org} - pasaulinio lygio rodiklių duomenų bazių rinkinys, turintis 69 skirtingas duomenų bazes, pavyzdžiui - ,,World development indicators'',
,,Gander statistics'' ir kitus\cite{databank-stats}.

\subsection{Charakteristikos}

Apibrėžime minėjau, kad rodiklių duomenis yra didelių duomenų tipas, todėl galime jiems pritaikyti didelių duomenų charakteristikas ir apsibrėžti, kurios iš jų 
mums sudaro daugiausiai problemų. Šie iššūkiai apibrėžiami Gartner's Doug Laney pristatytu 3V modeliu\cite{laney20013d}, kuris veliau buvo papildytas Bernard Marr iki 5V modelio\cite{marr2014big}:
\begin{itemize}
    \item Tūris (angl. Volume). Apibrežia generuojamų duomenų kiekius. Didelių duomenų atveju yra šnekama apie duomenų kiekius, kuriuos yra sudetinga arba neįmanoma saugoti 
    ir analizuoti tradicinėmis duomenų bazių technologijomis. Rodiklių duomenų kiekiai dažniausiai nesudaro problemos saugojant, tačiau didelė problema yra rodiklių duomenų analizė, 
    kadangi tuos pačius duomenis reikia apdoroti pagal neapribotą skaičių skirtingų rodiklių.
    \item Greitis (angl. Velocity). Apibrežia greitį, kuriuo nauji duomenis yra generuojami. Rodiklių duomenų atveju, tai yra labai svarbu, kadnagi nauji duomenis, kurie gali 
    tikti skirtingiems rodikliams yra generuojami visais laikais.
    \item Įvairovė (angl. Variety). Apibrežia duomenų tipus. Duomenys gali būti: strukturizuoti, nestrukturizuoti arba dalinai strukturizuoti\cite{zikopoulos2011understanding}. 
    Rodiklių duomenis dažinau yra strukturizuoti, todėl tai nėra aktualus iššūkis.
    \item Tikrumas (angl. Veracity). Apibrežia duomenų tesingumą ir kokybę. Pavyzdžiui, jeigu analizuotume ,,Twitter'' socialinio tinklo žinučių turinį gautume daug gramatikos klaidų, naujadarų, slengo. 
    Statistinio departamento atveju duomenys visada bus tvarkingi, kadangi tai dažniausiai yra duomenys surinkti iš dokumentų ir apklausų, o ne laisvo įvedimo.
    \item Vertė (angl. Value). Apibrežia duomenų ekonominę vertę. Rodiklių duomenys yra labai vertingi įstaigoms, nes dažniausiai tos įstaigos užsiema tik rodiklių duomenų kaupimų ir analizė, iš techninės pusės
    ši charakteristika yra svarbi iš tos pusės, kad duomenų apdorojimo ir kaupimo sprendimai labai stipriai daro įtaką įstaigos, kaupiančios rodiklių duomenis, ekonomikai. Taip pat šių duomenys ir jų 
    analizė turi būti pasiekiama be prastovos laiko.
\end{itemize}
    Pagal apibrėžtas charakteristikas matome, kad pagrindiniai rodiklių duomenų iššūkiai yra tūris, greitis ir vertė. Todėl mūsų bandomas sprendimas turi galėti greitai susidoriti su dideliu kiekių 
skirtingo pobūdžio duomenų ir turi sugebėti greitai atvaizduoti pokyčius atsiradus naujiems duomenims, taip pat turi būti įmanoma šį sprendimą paleisti į realią aplinką nepertraukiant įstaigos veiklą.

\section{Rodiklių duomenų apdorojimo tipai}


\subsection{Reliacinės duomenų bazės duomenų apdorojimas}

Aprašoma kaip generacija vyktu stored proceduros budu

\subsection{Srautinis duomenų apdorojimas}

Aprašoma kaip vyktų srautinis duomenu apdorojimas su schema.

\subsection{Kiti duomenų apdorojimo tipai}

\subsubsection{Paketinis duomenų apdorojimas}

Paketinis duomenų apdorojimas (angl. Batch processing) - didelių duomenų apdorojimo tipas, kai surinktas didelis duomenų kiekis nėra apdorojamas iš karto, o sukaupiamas ir vėliau apdorojamas visas iš karto. 
Mūsų reliacinės duomenų bazės apdorojimas naudoją primityvią paketinio duomenų apdorojimo versiją. Daug geriau yra žinomas kita paketinio duomenų apdorojimo architektūra - ,,Apache Hadoop''. Ši architektūra
naudoja ,,Google'' sukurtą ,,Map-Reduce'' konceptą\cite{dean2008mapreduce}, kuris dideli duomenų kiekį suskaido į mažus rinkinius ir daro apdorojima paraleliai ant visų rinkinių.\cite{batchProcessing} Ši architektūra mums netinka, nes mes norime rezultatą 
gauti tik įdėję naujų duomenų.

\subsubsection{Lambda architektūra}

Lambda architektūra - didelių duomenų apdorojimo architektūra naudojanti srautinio ir paketinio apdorojimo architektūrą. Su šią architektūra bandoma subalansuoti latencija, pralaidumą ir klaidų tolerancija, naudojant 
paketini apdorojimą tiksliems ir išsamiems duomenų pakentams ir tuo pačiu metu naudoti srautinį apdorojima greitai analizei\cite{hasani2014lambda}. Tačiau ši architektūra turi vieną trukūma - norint naudoti šią architektūrą
reikia palaikyti dvi skirtingas architektūras, kad kodas būtų atnaujinamas abiejose architektūrose vienu metu\cite{kreps2014questioning}. Relaus pavyzdys tokios architektūros būtų: Kafka žinučių sistema siunčianti duomenis į ,,Apache Storm'', 
kur duomenis apdorojami srautiškai, ir ,,Apache Hadoop'', kur duomenis apdorojami paketiškai, ir rezultatus saugant skirtingose duomenų bazės lentelėse.

\subsubsection{Kappa architektūra}

Kappa architektūra - tai supaprastinta lambda architektūra, kuri vietoj paketinio apdorojimo naudoja papildomą srautinio adorojimo systemą. Pavyzdžiui yra vienas srautinio apdorojimo darbas, kuris veikia visą laiką
ir atvaizduoja duomenis gyvai, ir yra kitas darbas kuris paleidžiamas kas kažkiek laiko, kuris susirenka per visą tą laiką susikaupusius duomenis ir praleidžia per save srautu. Kadangi kodai yra vienodi, tai nebeatsiranda 
anksčiau minetos problemos su skirtingų sistemų palaikymu\cite{kreps2014questioning, kappa}.

\section{Testinių duomenų generatorius}

\subsection{Apibūdinimas}

Papasakoti čia apie Kafka

\subsection{Paskirtis ir panaudojimo būdas}

Papasakoti čia apie throtlinimo testavimą

\section{Sprendimo naudojančio relaicinę duomenų bazę pralaidumo testas}

\subsection{Apibūdinimas}

\subsection{Testavimo rezultatai}

\section{Srautinio apdorojimo architaktūros}
Šiame skyriuje palyginsiu skirtingas srautinio apdorojimo architektūrų privalumus ir trūkumus ir pasirinksiu vieną su kuria sukursiu sprendimą rodiklių duomenų apdorojimui.
\subsection{,,Apache Storm''}

\subsection{,,Apache Spark''}

\subsection{,,Apache Flink''}

\section{Srautinės architektūros sprendimo pralaidumo testas}

\subsection{Apibūdinimas}

\subsection{Rezultatas}

\section{Experimento apibendrinimas}

\subsection{Rezultatai}

\subsection{Eksperimento išvados}

\sectionnonum{Rezultatai ir išvados}
Darbo rezultatai:
\begin{itemize}
    \item Išnagrinėti rodiklių duomenų analizės būdai pagal jų privalumus ir trūkumus.
    \item Sukurtas testinių duomenų generatorius, kurio pagalba buvo vykdomas pralaidumo testavimas.
    \item Atliktas pralaidumo testas Micorsoft SQL Express duomenų bazei su sukurtu testiniu duomenų generatorium ir tarpine Python aplikacija.
    \item Išanalizuoti skirtingos srautinio apdorojimo architektūros ir pasirinkta tinkamiausia sprendimo kurimui.  
    \item Sukurtas sprendimas su pasirinkta srautinio apdorojimo architektūra.
    \item Atliktas pralaidumo testas sukurtam srautinio apdorojimo sprendimui.
    \item Palyginti sprendimų pralaidumo testų rezultatai.   
\end{itemize}

Darbo išvados:
\begin{itemize}
\item 
\item
\end{itemize}

\printbibliography[heading=bibintoc] 

\end{document}
